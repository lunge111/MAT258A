\documentclass[12pt,a4paper]{article}
\usepackage{multirow}
\usepackage{bm}
\usepackage{AMSFONTS}
\usepackage{amssymb}
\usepackage{latexsym}
\usepackage{graphicx}
\usepackage{subfigure}
\usepackage{ hyperref}
\usepackage[style=numeric]{biblatex}

\addbibresource{bibliography.bib}

\textwidth 6.5in
\textheight 9in
\topmargin 1pt
\linespread{1.5}
\oddsidemargin 0pt
\begin{document}
\title{\huge{HW6 }}

\author{Yilun Zhang}
\newtheorem{coro}{\hskip 2em Corollary}[section]
\newtheorem{remark}[coro]{\hskip 2em Remark}
\newtheorem{propo}[coro]{\hskip 2em  Proposition}
\newtheorem{lemma}[coro]{\hskip 2em Lemma}
\newtheorem{theor}[coro]{\hskip 2em Theorem}
\newenvironment{prf}{\noindent { proof:} }{\hfill $\Box$}
\date{\today}
\maketitle
\section*{Exercise 1}
\subsection*{a}
True. In this case, the null space of $A$ has one dimension. Suppose $x_1\in null(A)$. $x_0$ is an element of $P$. Any elements of $P$ is of the form $x_0+cx_1,c\in\mathbb{R}$. Then $P$ is constrained in a line and can't have 2 basic feasible point.
\subsection*{b}
False. Consider minimize $c$, $c$ is a constant, subject to $x\ge0$, the optimal solution is $x\in [0,+\infty)$ is unbounded.
\subsection*{c}
False. Consider the example of (b), any feasible solution is optimal.
\subsection*{d}
True. If $x_1$ and $x_2$, are optimal, any convex combination of them are optimal solution.'
\subsection*{e}
False. Consider min $x_1$ subject to $x_1=0,x_2\ge0$ the optimal solution is $\{0\}\times[0,\infty)$ is infinitely many, but only has one optimal BFS.
\subsection*{f}
False. Consider max $|x_1-0.5|=max\{x_1-0.5x_3,-x_1+0.5x_3\}$ subject to $x_3=1,x_1+x_2=1,x_1,x_2,x_3\ge0$. The unique optimal solution is (0.5, 0.5, 1), but is not basic solution.
\section*{Exercise 2}
\subsection*{a}
False, $A$ has full rank, then BFS is non-degenrated. When $x_j$ entered, the cost is strickly decreased since the solution is moving along a feasible direction chosen is based on $c_j<0$, So the cost must change.
\subsection*{b}
Because a variable can be entered only if the reduced cost is negative. A variable is leaving iff the coresponding reduced cost is nonnegative. So in next iteration, the variable will not reenter.
\subsection*{c}
 False. Consider the problem min $-x_1 - 2x_2$ such that $x_1 + x_2 \le 1$, and $x_1, x_2 \ge 0$. The transformed problem is $x_1 + x_2 +x_3= 1$, and $x_1, x_2 , x_3\ge 0$. If we start with $x_3$ and (0,0,1) with cost function equal to 0. Then in next interation the argorithm moves to (1,0,0), and the cost function is -1. In the very next interation, the argorithm moves to (0,1,0) with cost function -2. This is a counterexample is the statement.
 
\subsection*{d}
 False. Consider the min $x_2$ with the same constraint in $(b)$. The optomal solution is $x_2=0$ and $x_1\in [0,1]$. But it has two linear independent BFS (1,0,0) and (0,0,1).
 
\subsection*{e}
 True. Directly follows Thm 2.4 of Bertsemas and Tistsiklis,
\emph{ Introduction to Linear Optimization} (1997)
\end{document} 