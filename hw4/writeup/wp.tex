\documentclass[12pt,a4paper]{article}
\usepackage{multirow}
\usepackage{bm}
\usepackage{AMSFONTS}
\usepackage{amssymb}
\usepackage{latexsym}
\usepackage{graphicx}
\usepackage{subfigure}
\usepackage{ hyperref}
\usepackage{biblatex}
\bibliography{bibliography}

\textwidth 6.5in
\textheight 9in
\topmargin 1pt
\linespread{1.5}
\oddsidemargin 0pt
\begin{document}
\title{\huge{HW4 }}


\author{Yilun Zhang}
\newtheorem{coro}{\hskip 2em Corollary}[section]
\newtheorem{remark}[coro]{\hskip 2em Remark}
\newtheorem{propo}[coro]{\hskip 2em  Proposition}
\newtheorem{lemma}[coro]{\hskip 2em Lemma}
\newtheorem{theor}[coro]{\hskip 2em Theorem}
\newenvironment{prf}{\noindent { proof:} }{\hfill $\Box$}
\date{\today}
\maketitle
\section*{Exercise 1}
Method 1, let $f(x)=dx-1$, so the solution of $f(x)=0$ is $1/d$. 
The matlab code is \begin{verbatim}
function [x l] = newroot(x0,a, maxiter, tol)
fold = a*x0-1;
fnew = fold-1;
x=x0;
iter=0;
while( abs(fnew-fold)>tol && iter< maxiter)
fold = fnew;
iter=iter+1;
x=x-(a*x-1)/a;
fnew=a*x+1;
end
l=iter;
end
\end{verbatim}
Start value of both 0.3 and 1 both converge in 1 step, this is because\[
x_{k+1}=x_k-\frac{ex_k-1}{e}=\frac{1}{e}
\]
Method 2, let $g(x)=1/x-d$, the update steps is \[
x_{k+1}=x_k+(1/x_k-d)x_k^2=2x_k-dx_k^2
\]
Then if the start value is 0.3 it will converge to the true value, if the start value is 1, it will not converge.
\section*{Exercise 2}
\subsection*{a}
\begin{eqnarray*}
x_{k+1}&=&x_k-\frac{x_k^q}{qx^{q-1}_k}\\
&=& x_k(1-\frac{1}{q})
\end{eqnarray*}
Therefore, it's Q-linearly converge.
\subsection*{b}
Let $f(x)=||x||_2^\beta$\[
\nabla f=\beta ||x||_2^{\beta-2} x
\]

\[
\nabla^2 f=\beta||x||_2^{\beta-4}(||x||_2^{2}I+(\beta-2)xx')
\]
Because $(A+CBC')^{-1}= A^{-1}-A^{-1}C(B^{-1}+C'A^{-1}C)^{-1}C'A^{-1}$
\[
(\nabla^2 f)^{-1}=\frac{1}{\beta||x||_2^{\beta-2}}(I-\frac{\beta-2}{\beta-1}\frac{1}{||x||_2^2}xx')
\]

The pure Newton method: \begin{eqnarray*}
x_{k+1}&=&x_k-(\nabla^2 f(x_k))^{-1}\nabla f(x_k)\\
&=& x_k-\frac{1}{\beta||x_k||_2^{\beta-2}}(I-\frac{\beta-2}{\beta-1}\frac{1}{||x_k||_2^2}x_kx_k')\beta ||x_k||_2^{\beta-2} x_k\\
&=&\frac{\beta-2}{\beta-1}x_k
\end{eqnarray*}

Then, $x_k$ converges Q-linearly when $\beta>3/2$ globally. When $\beta=2$, it converges in 1 step. When $\beta \le 1$, it will not converge for start point $x_0\neq0$.

\subsection*{c}
The Newton method: \begin{eqnarray*}
	x_{k+1}&=&x_k-\alpha_k(\nabla^2 f(x_k))^{-1}\nabla f(x_k)\\
	&=& x_k-\alpha_k\frac{1}{\beta||x_k||_2^{\beta-2}}(I-\frac{\beta-2}{\beta-1}\frac{1}{||x_k||_2^2}x_kx_k')\beta ||x_k||_2^{\beta-2} x_k\\
	&=&(1-\frac{\alpha_k}{\beta-1})x_k
	\end{eqnarray*}
	
	The Armijo condition: $f(x_k+\alpha_kp_k)\le f(x_k )+c_1\alpha_k[\triangledown f(x_k)]p_k $. This implies
	\[
	\left[1-|1-\frac{\alpha_k}{\beta-1}|^\beta \right]||x_k||_2^\beta \le c_1\alpha_k \frac{\beta}{\beta-1} ||x_k||_2^\beta
	\]
	
	as long as $||x_k||_2$ is not zero, we can see $\alpha_k$ doesn't dependent on $k$, then let $\alpha_k\equiv\alpha$ 
	
	where $\alpha$ satisfies\[
	\left[1-|1-\frac{\alpha_k}{\beta-1}|^\beta \right]\le c_1\alpha_k \frac{\beta}{\beta-1} 	
	\]
	
	Then, $x_k$ converges Q-linearly when $|1-\frac{\alpha}{\beta-1}|<1$ globally. When $\frac{\alpha}{\beta-1}=1$, it converges in 1 step. When $\beta \le 1$, it will not converge for start point $x_0\neq0$, because $|1-\frac{\alpha_k}{\beta-1}|$ always larger than 1. when $\alpha>0$ and $\beta>1$, it will Q- linearly converge.


\section*{Excercise 3}
Prove this by induction.

For $k=0,i=0$, need to show $D^1q^0=p^0$
\[
D^1q^0=(D^0+\frac{y^0y^{0'}}{q^{0'}y^0})q^0=D^0q^0+\frac{y^0y^{0'}q^0}{q^{0'}y^0}=p^0
\]

Suppose k-1 hold,\[
D^{k+1}q^i=(D^{k}+\frac{y^ky^{k'}}{q^{k'}y^k})q^i=D^{k}q^i+\frac{y^ky^{k'}q^k}{q^{k'}y^k}=p^i
\]
if $i<k$ \[
LHS= D^{k}q^i+\frac{y^ky^{k'}q^k}{q^{k'}y^k}=p^i+\frac{1}{q^{k'}y^k}(p^{k'}q_i-q^{k'}p^i)
\]
since $Qp^i=q^i$, and $Q$ is positive definit thus invertible. $p^{k'}q_i-q^{k'}p^i=0$. So the LHS=$p^i$

if $i=k$\[
D^{k+1}q^k=(D^k+\frac{y^ky^{k'}}{q^{k'}y^k})q^k=D^kq^k+\frac{y^ky^{k'}q^k}{q^{k'}y^k}=p^k
\]
therefore , we have $D^{k+1}q^i=p^i$

Since $D^nq^i=p^i$ and $Q^{-1}q^i=p^i$ for $i=0,1,\cdots,n-1$. Let $X=[q^0, q^1, \cdots,q^{n-1}]$. Because $\{q^i\}$ are linearly independent. X is full rank. $(D^n-Q^{-1})X=0$ implies $D^n=Q^{-1}$.

\section*{Exercide 4}
I already implemented BFGS in last homework, the matlab code:
\begin{verbatim}
function [x,fc,itc] = newton(obj,i,maxit,tol,qusi,eps)
c=0.0001;
x0=p00_start(i,p00_n(i));
n=p00_n(i);
[fc,gc,hc]=obj(i,x0); 
[P,D]=eig(hc);
if(min(diag(D))<eps)
D=max(D,eps*eye(n));
hc=P*D*P';
end
g0=gc;
xc=x0;
itc=1;
H=inv(hc);
while(norm(gc) > tol*norm(g0) & itc <= maxit)
p=-H*gc;
alpha=1.0; xt=xc+alpha*p; ft=obj(i,xt);
fg= fc + c*alpha*(p'*gc);
cout=1;
while(ft > fg) % check Armijo condition
alpha=alpha*0.9;
fg= fc + c*alpha*(gc'*p);
xt=xc+alpha*p;
ft=obj(i,xt);
cout=cout+1;
if(cout>20)
break
end
end
xc=xt;
go=gc;
[fc,gc,hc]=obj(i,xc);
itc=itc+1;
if(qusi)
s=alpha*p;
y=gc-go;
pho=1/(y'*s);
H=(eye(n)-pho*s*y')*H*(eye(n)-pho*y*s')+pho*s*s';

else
[P,D]=eig(hc);
if(min(diag(D))<eps)
D=diag(D);          
D(D<=1e-8)=eps;       
D=diag(D); 
hc=P*D*P'
H=inv(hc);
end
end   
end
x=xc;


end
\end{verbatim}
Table 1 is the result of pure Newton method(with eigen value correction), Table 2 shows result of quasi newton.

\begin{table}
	\begin{tabular}{|c|c|c|} \hline
		prob ID & $f^*$ & number of iter\\		\hline
		1& 4.46500818473627	&20001\\
		2& 0.242677406804875&	412\\
		3 & 1.12793276962393e-08	&4\\
		4& 0.0032230846252832&	20001\\
		5 &3.97089247824700e-14&	6\\
		6 &6.37118858804940e-12&	45\\
		7 &0.00228767657879530	&3076\\
		8 &2.49997654797500e-06	&59\\
		9 &0.489395214700814	&16\\
		10& 25364766502.5965	&20001\\
		11 &85822.2019034065	&251\\
		12 &8.39437213309053e-20&	13\\
		13 &8.42718143579277e-16&	13\\
		14 &4.93038065763132e-32&	2\\
		15 &1.63898347747199e-06&	14859\\
		16 &1.81352665939531e-08&	20001\\
		17 &7.79452679362206	&20001\\
		18 &5.93799296903267e-14&	13	\\	\hline
	\end{tabular}
	\caption{result of Newton method}
\end{table}


\begin{table}
	\begin{tabular}{|c|c|c|} \hline
		prob ID & $f^*$ & number of iter\\		\hline
	1&	4.75669872116614e-07&	36\\
	2&	NaN	&3\\
	3&	1.12793283165719e-08&	3\\
	4&	8.23648565011059e-10&	203\\
	5&	NaN	&3\\
	6&	3.52702079374637e-18&	7\\
	7&	0.00228767029059872&	20\\
	8&	2.49997507553623e-06&	8\\
	9&	0.489395214700777&	8\\
	10&	NaN&	8\\
	11&	85822.2162472798&	18\\
	12&	32.8349996345640&	7\\
	13&	1.80370537607437e-13&	5\\
	14&	4.93038065763132e-32&	2\\
	15&	1.67963844460425e-05&	15\\
	16&	NaN&	5\\
	17&	7.87479283486014&	13\\
	18&	0.444444444430489&	203\\ \hline
	\end{tabular}
	\caption{result of quasi Newton method}
\end{table}
 We can see all problems converge but number 2, 5, 10, 16 return a NA. I guess this is because $\rho=1/y's$, when $y$ and $s$ are both very small, it will cause overflow in calculation in Matlab.
\end{document} 